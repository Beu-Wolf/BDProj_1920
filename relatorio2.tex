\documentclass[12pt]{report}
\usepackage[textwidth=17cm, margin=2cm]{geometry}
\usepackage[utf8]{inputenc}
\usepackage[T1]{fontenc}
\usepackage{graphicx}
\usepackage{enumitem}
\usepackage{multicol}

\begin{document}

    \begin{titlepage}
        \begin{center}

            \vspace*{\fill}
            \Huge
            \textbf{Projeto de Bases de Dados - Parte 2}
            
            \vspace*{\fill}

            \Large
            \textbf{Grupo 27} \\
            89427 - Daniel Seara - 33,3\% - 3 horas \\
            89399 - Afonso Gonçalves - 33,3\% - 3 horas \\
            89496 - Marcelo Santos - 33,3\% - 3 horas \\

            \bigskip
            \textbf{Turno:} 4ª Feira 9h30 - Lab 8\\ \textbf{Professor:} Duarte Galvão
        
        \end{center}
    \end{titlepage}






\Large
\textbf{Modelo Relacional:}\\

\normalsize
\vspace{2mm}

\begin{multicols}{2}
Item(\underline{id}, descrição, localização, coordenadas)
    \begin{itemize}
    \item coordenadas: FK(Local Público)
    \end{itemize}

\vspace{5mm}

%Add integrity constraint for disjointness and Total participation

Utilizador(\underline{email}, password)

Utilizador Regular(\underline{email})
    \begin{itemize}
    \item email: FK(Utilizador)
    \end{itemize}


Utilizador Qualificado(\underline{email})
    \begin{itemize}
    \item email: FK(Utilizador)
    \end{itemize}

\vspace{5mm}

%Add integrity constraint for disjointness and Total participation
% and RI-1 and RI-2 from project description


Anomalia(\underline{id}, descrição, ts, zona, imagem, língua)


Anomalia Redação(\underline{id})
    \begin{itemize}
    \item id: FK(Anomalia)
    \end{itemize}


Anomalia Tradução(\underline{id}, língua2, zona2)
    \begin{itemize}
    \item id: FK(Anomalia)
    \end{itemize}


\vspace{5mm}


Local Público(\underline{coordenadas})


\vspace{5mm}


Proposta de Correção(\underline{nro, email, anomalia\_id}, datahora, texto)
    \begin{itemize}
	    \item email: FK(Utilizador Qualificado)
	    \item anomalia\_id: FK(Incidência)
    \end{itemize}



\vspace{5mm}


cria(\underline{nro}, U\_email)
    \begin{itemize}
    \item nro: FK(Proposta Correção)
    \item U\_email: FK(Utilizador Qualificado)
    \end{itemize}   


\vspace{5mm}

%Add RI-3 from project decription


duplicado(\underline{id1, id2})
    \begin{itemize}
	    \item id1, id2: FK(Item(id))
    \end{itemize}


\vspace{5mm}

incidência(\underline{anomalia\_id}, email, item\_id)
    \begin{itemize}
	    \item anomalia\_id: FK(Anomalia(id))
	    \item email: FK(Utilizador)
	    \item item\_id: FK(Item(id))
    \end{itemize}


\vspace{5mm}

\end{multicols}

\Large
\textbf{Restrições de Integridade: }\\

\normalsize
\begin{enumerate}
    \item RI-1 - As \textit{zonas} não se podem sobrepor
    \item RI-2 - As \textit{línguas} não podem ser iguais
    \item RI-3 - Um \textit{item} não pode ser duplicado de si próprio
    \item RI-4 - Cada \textit{U\_email} tem de estar em \textit{Utilizador Qualificado} ou em \textit{Utilizador Regular}, mas não em ambos
    \item RI-5 - Cada \textit{A\_id} tem de estar em \textit{Anomalida Redação} ou em \textit{Anomalia Tradução}, mas não em ambos
    
\end{enumerate}


\end{document}
