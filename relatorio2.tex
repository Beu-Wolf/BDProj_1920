\documentclass[12pt]{report}
\usepackage[textwidth=17cm, margin=2cm]{geometry}
\usepackage[utf8]{inputenc}
\usepackage[T1]{fontenc}
\usepackage{graphicx}
\usepackage{enumitem}
\usepackage{multicol}
\usepackage{amsmath}

\begin{document}

    \begin{titlepage}
        \begin{center}

            \vspace*{\fill}
            \Huge
            \textbf{Projeto de Bases de Dados - Parte 2}
            
            \vspace*{\fill}

            \Large
            \textbf{Grupo 27} \\
            89427 - Daniel Seara - 33,3\% - 3 horas \\
            89399 - Afonso Gonçalves - 33,3\% - 3 horas \\
            89496 - Marcelo Santos - 33,3\% - 3 horas \\

            \bigskip
            \textbf{Turno:} 4ª Feira 9h30 - Lab 8\\ \textbf{Professor:} Duarte Galvão
        
        \end{center}
    \end{titlepage}






\Large
\textbf{Modelo Relacional:}\\

\normalsize
\vspace{2mm}

\begin{multicols}{2}
Item(\underline{id}, descrição, localização, coordenadas)
    \begin{itemize}
    \item coordenadas: FK(Local Público)
    \end{itemize}

\vspace{5mm}
%Add integrity constraint for disjointness and Total participation

Utilizador(\underline{email}, password)$^{[RI-4]}$

\vspace{5mm}

Utilizador Regular(\underline{email})
    \begin{itemize}
    \item email: FK(Utilizador)
    \end{itemize}


Utilizador Qualificado(\underline{email})
    \begin{itemize}
    \item email: FK(Utilizador)
    \end{itemize}

\vspace{5mm}

%Add integrity constraint for disjointness and Total participation
% and RI-1 and RI-2 from project description


Anomalia(\underline{id}, descrição, ts, zona, imagem, língua)$^{[RI-1,2,5]}$


Anomalia Redação(\underline{id})
    \begin{itemize}
    \item id: FK(Anomalia)
    \end{itemize}


Anomalia Tradução(\underline{id}, língua2, zona2)$^{[RI-1,2]}$
    \begin{itemize}
    \item id: FK(Anomalia)
    \end{itemize}


\vspace{5mm}


Local Público(\underline{coordenadas})

\vspace{5mm}

Proposta de Correção(\underline{nro, email}, datahora, texto)$^{[RI-6]}$
    \begin{itemize}
	    \item email: FK(Utilizador Qualificado)
	    \item anomalia\_id: FK(Incidência)
    \end{itemize}



\vspace{5mm}

Correção(\underline{nro, email, anomalia\_id})
\begin{itemize}
	\item nro, email: FK(Proposta de Correção)
	\item anomalia\_id: FK(Incidência)
\end{itemize}

%Add RI-3 from project decription


Duplicado(\underline{id1, id2})$^{[RI-3]}$
    \begin{itemize}
	    \item id1, id2: FK(Item(id))
    \end{itemize}


\vspace{5mm}

Incidência(\underline{anomalia\_id}, email, item\_id)
    \begin{itemize}
	    \item anomalia\_id: FK(Anomalia(id))
	    \item email: FK(Utilizador)
	    \item item\_id: FK(Item(id))
    \end{itemize}


\vspace{5mm}

\end{multicols}

\Large
\textbf{Restrições de Integridade: }\\

\normalsize
\begin{enumerate}
    \item RI-1 - As \textit{zonas} não se podem sobrepor
    \item RI-2 - As \textit{línguas} não podem ser iguais
    \item RI-3 - Um \textit{item} não pode ser duplicado de si próprio
    \item RI-4 - Cada \textit{email} tem de estar em \textit{Utilizador Qualificado} ou em \textit{Utilizador Regular}, mas não em ambos
    \item RI-5 - Cada \textit{id} de \textit{Anomalia} tem de estar em \textit{Anomalia Redação} ou em \textit{Anomalia Tradução}, mas não em ambos
    \item RI-6 - Cada \textit{nro} de \textit{Proposta de Correção} tem de estar em \textit{Correção}
    
\end{enumerate}

\newpage
\Large
\textbf{Álgebra Relacional: }\\

\begin{enumerate}
	\item $\pi_{texto}(\sigma_{datahora = 2019}(\text{Proposta de Correção}))$
	\item $\pi_{texto}(\text{Proposta de Correção}) \cup \pi_{\text{descrição}}(\sigma_{\text{(língua = Português)}}(Anomalia)) \cup
		\pi_{\text{descrição}}(Item)$
	\item $\pi_{password}(\sigma_{datahora = 1/10/2019})(\text{Utilizador}\bowtie\text{Proposta de Correção})$
	\item $\pi_{email}(\sigma_{datahora = 1/10/2019 - 20:00}(\text{Proposta de Correção}))$
	\item $R_{1} \gets id_{1}G_{count(id_{1}) \text{ as id\_count}}(Duplicado),$\\
		$ \pi_{id_{1}}(R_{1} - \pi_{A.id_{1}, A.id\_count}(\sigma_{A.id\_count < B.id\_count}(\rho_{A}(R_{1}) \times \rho_{B}(R_{1}))))$
\end{enumerate}

\Large
\textbf{SQL}

\begin{enumerate}
	\item SELECT \textit{texto} \\
		FROM \textit{Proposta de Correção} \\
		WHERE \textit{datahora} = 2019;
	\item SELECT texto \\
		FROM \textit{Proposta de Correção} \\
		UNION \\
		SELECT \textit{descrição} \\
		FROM \textit{Anomalia} \\
		WHERE \textit{língua = Português} \\
		UNION \\
		SELECT \textit{descrição} \\
		FROM \textit{Item};
\end{enumerate}

\end{document}
