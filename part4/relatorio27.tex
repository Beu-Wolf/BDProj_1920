\documentclass[12pt]{report}
\usepackage[textwidth=17cm, margin=2cm]{geometry}
\usepackage[utf8]{inputenc}
\usepackage[T1]{fontenc}
\usepackage{graphicx}
\usepackage{enumitem}
\usepackage{multicol}
\usepackage{amsmath}
\usepackage{listings}

\begin{document}

    \begin{titlepage}
        \begin{center}

            \vspace*{\fill}
            \Huge
            \textbf{Projeto de Bases de Dados - Parte 3}

            \vspace*{\fill}

            \Large
            \textbf{Grupo 27} \\
            89427 - Daniel Seara - 33,3\% - 5 horas \\
            89399 - Afonso Gonçalves - 33,3\% - 5 horas \\
            89496 - Marcelo Santos - 33,3\% - 5 horas \\

            \bigskip
            \textbf{Turno:} 4ª Feira 9h30 - Lab 8\\ \textbf{Professor:} Duarte Galvão

        \end{center}
    \end{titlepage}

    \Large
    \textbf{Restrições de Integridade}\\
    \normalsize
    \par Foram usados triggers para implementar as Restrições de Integridade, apresentados de seguida:
    \small
    \begin{verbatim}
    -- RI-1
    create or replace function
    check_overlap_proc() returns trigger
    as $$
        declare
            zona1   box;
        begin
            select zona into zona1
            from anomalia where id=new.id;
            
            if zona1 && new.zona2 then
                raise exception
                'As zonas da anomalia % não se podem intersetar.', new.id;
            end if;
            return new;
        end;
    $$ language plpgsql;
    
    create trigger check_overlap
    after insert on anomalia_traducao 
    for each row execute procedure check_overlap_proc();
    
    -- RI-4
    create or replace function
    check_completeness_proc() returns trigger
    as $$
        declare 
            inRegular   boolean;
            inQualified boolean;
        begin
            select exists(
                select 1
                from utilizador_regular
                where email = new.email
            ) into inRegular;
    
            select exists(
                select 1
                from utilizador_qualificado
                where email = new.email
            ) into inQualified;
        
            if inRegular and inQualified then
                raise exception
                'Utilizador % não pode estar em utilizador_regular e\
                utilizador qualificado simultaneamente', new.email;

            elseif (not inRegular and not inQualified) then
                raise exception
                'Utilizador % tem de estar em utilizador_regular\
                ou utilizador_qualificado', new.email;
            end if;
            return new;
        end;
    $$ language plpgsql;
    
    create constraint trigger check_completeness
    after insert on utilizador
    deferrable initially deferred
    for each row execute procedure check_completeness_proc();
    
    
    -- RI-5
    create or replace function
    check_user_qualificado_proc() returns trigger
    as $$
        declare
            inRegular boolean;
        begin
            select exists(
                select 1
                from utilizador_regular
                where email = new.emai
            ) into inRegular;
    
            if inRegular then
                raise exception
                'Utilizador % já está em utilizador_regular', new.email;
            end if;
            return new;
        end;
    $$ language plpgsql;
    
    create trigger check_user_qualificado
    after insert on utilizador_qualificado
    for each row execute procedure check_user_qualificado_proc();
    
    
    --RI-6
    create or replace function
    check_user_regular_proc() returns trigger
    as $$
        declare
            inQualified boolean;
        begin
            select exists(
                select 1
                from utilizador_qualificado
                where email = new.email
            ) into inQualified;
            
            if inQualified then
                raise exception
                'Utilizador % já está em utilizador_qualificado', new.email;
            end if;
            return new;
        end;
    $$ language plpgsql;
    
    create trigger check_user_regular
    after insert on utilizador_regular
    for each row execute procedure check_user_regular_proc();
    \end{verbatim}
    \normalsize
    
    \Large
    \textbf{Índices TODO:}% Format this to be pretty pls
    \normalsize \\
    \par 1.1
    \par Um índice na tabela proposta\_de\_correcao com o atributo data\_hora como chave de procura é um índice \textit{unclustered}: No pior dos casos, seria necessário percorrer todas as páginas do disco pela qual a tabela se distribui para se encontrar um resultado.

    \par Tendo em conta que esta query devolve mais de 10\% do total de registos da tabela e que o custo associado à indexação (percorrer a B+-Tree, juntamente com o custo associado a carregar de disco todas as páginas de memória necessárias para encontrar o resultado pretendido) ultrapassa o custo de apenas percorrer a tabela toda, não compensa criar um índice para esta pesquisa.

    \par Esta ideia é reforçada pela bibliografia da cadeira:
    \begin{quotation}
        "As a rule of thumb, it is probably cheaper to simply scan the entire table (instead of using an unclustered index) if over 5\% of the tuples are to be retrieved." ~ TODO:% citar do manual de BD versao 3, secção 12.3.1. Usar bibitem?
    \end{quotation}
    % check: a quote do livro serve para qualquer tipo de indice?
    
    
    \par 1.2% Format this to be pretty pls
    \par Seguindo o raciocínio usado na questão anterior e tendo em conta que, neste caso, o query devolve menos de 0.001\% do total de registos da tabela, é justificável a utilização de índices.
    
    \par Como o filtro de pesquisa é um intervalo de valores, decidiu-se implementar um indice com B+-Tree, uma vez que é o mais adequado para este tipo de procura: Ao contrário dos Hash Indices, a B+-Tree mantém os índices ordenados por chave. Deste modo é possível filtrar um intervalo destes mesmos índices, encontrando o índice com o valor mínimo para o intervalo e explorando o conjunto ordenado dos índices sucessores, enquanto estes ainda estiverem dentro do intervalo pretendido.
    \par O índice foi aplicado sobre o atributo \textit{{data\_hora}} da tabela \textit{proposta\_de\_correcao}\\
    
    \par 2
    \par Neste query está a ser procurado um registo pela sua \textit{primary key}. Como o pgsql cria automaticamente um índice para as chaves primárias de cada tabela, não será necessário criar um novo índice.\\
    
    \par 3.1
    \par Pelas mesmas razões apresentadas em 1.1, este caso não é o melhor para se usar índices, acabando a sua implementação por ser mais dispendiosa do que uma procura simples.\\
    
    \par 3.2
    \par TODO \\
    
    
    \par 4
    \par 0
    
    \par Para minimizar o numero de blocos acedidos em disco neste \textit{query} decidiu-se implementar um índice que reduzisse o espaço de procura. Ao usar o índice múltiplo \textit{(timestamp, language)} numa B+-Tree na tabela \textit{anomalia} conseguimos minimizar o numero de acessos a disco.
    
    % [?] e preciso dizer como e que reduz?\\
    % [?] temos de mudar a ordem do where no query? passar para ts, language e tem\_an\_red?\\
    
    % [!] eh importante dizer que a ordem das chaves tem de ser a mesma da ordem de selecao do query
    % \\
    % \\
    \Large
    \textbf{Modelo Multidimensional}\\
    \normalsize
    \par Para criar um modelo Multidimensional, usou-se um esquema em estrela com as seguintes  dimensões:
    
    \begin{enumerate}[leftmargin=3\parindent]
        \item Utilizador (Quem criou a Anomalia)
        \item Tempo (Quando a Anomalia foi criada)
        \item Local (Onde a Anomalia foi criada)
        \item Língua (Em que linguagem está a Anomalia)
    \end{enumerate}

    \par De seguida apresenta-se o código usado para criar o Modelo Multidimensional
    \small \begin{verbatim}
    TODO: APENAS ESCREVER CODIGO QUANDO ESTIVER CHECKED
    \end{verbatim}\normalsize
    
    
    \Large
    \textbf{Data Analytics}\\
    \normalsize
    \par Para efetuar a análise pedida decidiu-se, por simplicidade, usar a instrução CUBE:
    \small \begin{verbatim}
    select U.tipo, L.lingua, T.dia_da_semana, count(tipo_anomalia) 
    from f_anomalia         natural join
         d_utilizador as U  natural join
         d_tempo as T       natural join 
         d_lingua as L 
    group by cube(U.tipo, L.lingua, T.dia_da_semana);
    \end{verbatim}\normalsize
\end{document}
