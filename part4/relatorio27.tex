\documentclass[12pt]{report}
\usepackage[textwidth=17cm, margin=2cm]{geometry}
\usepackage[utf8]{inputenc}
\usepackage[T1]{fontenc}
\usepackage{graphicx}
\usepackage{enumitem}
\usepackage{multicol}
\usepackage{amsmath}
\usepackage{listings}

\begin{document}

    \begin{titlepage}
        \begin{center}

            \vspace*{\fill}
            \Huge
            \textbf{Projeto de Bases de Dados - Parte 3}

            \vspace*{\fill}

            \Large
            \textbf{Grupo 27} \\
            89427 - Daniel Seara - 33,3\% - 5 horas \\
            89399 - Afonso Gonçalves - 33,3\% - 5 horas \\
            89496 - Marcelo Santos - 33,3\% - 5 horas \\

            \bigskip
            \textbf{Turno:} 4ª Feira 9h30 - Lab 8\\ \textbf{Professor:} Duarte Galvão

        \end{center}
    \end{titlepage}

    \Large
    \textbf{Restrições de Integridade}\\
    \normalsize
    \par Foram usados triggers para implementar as Restrições de Integridade, apresentados de seguida:
    \small
    \begin{verbatim}
    -- RI-1
    create or replace function
    check_overlap_proc() returns trigger
    as $$
        declare
            zona1   box;
        begin
            select zona into zona1
            from anomalia where id=new.id;
            
            if zona1 && new.zona2 then
                raise exception
                'As zonas da anomalia % não se podem intersetar.', new.id;
            end if;
            return new;
        end;
    $$ language plpgsql;
    
    create trigger check_overlap
    after insert on anomalia_traducao 
    for each row execute procedure check_overlap_proc();
    
    -- RI-4
    create or replace function
    check_completeness_proc() returns trigger
    as $$
        declare 
            inRegular   boolean;
            inQualified boolean;
        begin
            select exists(
                select 1
                from utilizador_regular
                where email = new.email
            ) into inRegular;
    
            select exists(
                select 1
                from utilizador_qualificado
                where email = new.email
            ) into inQualified;
        
            if inRegular and inQualified then
                raise exception
                'Utilizador % não pode estar em utilizador_regular e\
                utilizador qualificado simultaneamente', new.email;

            elseif (not inRegular and not inQualified) then
                raise exception
                'Utilizador % tem de estar em utilizador_regular\
                ou utilizador_qualificado', new.email;
            end if;
            return new;
        end;
    $$ language plpgsql;
    
    create constraint trigger check_completeness
    after insert on utilizador
    deferrable initially deferred
    for each row execute procedure check_completeness_proc();
    
    
    -- RI-5
    create or replace function
    check_user_qualificado_proc() returns trigger
    as $$
        declare
            inRegular boolean;
        begin
            select exists(
                select 1
                from utilizador_regular
                where email = new.emai
            ) into inRegular;
    
            if inRegular then
                raise exception
                'Utilizador % já está em utilizador_regular', new.email;
            end if;
            return new;
        end;
    $$ language plpgsql;
    
    create trigger check_user_qualificado
    after insert on utilizador_qualificado
    for each row execute procedure check_user_qualificado_proc();
    
    
    --RI-6
    create or replace function
    check_user_regular_proc() returns trigger
    as $$
        declare
            inQualified boolean;
        begin
            select exists(
                select 1
                from utilizador_qualificado
                where email = new.email
            ) into inQualified;
            
            if inQualified then
                raise exception
                'Utilizador % já está em utilizador_qualificado', new.email;
            end if;
            return new;
        end;
    $$ language plpgsql;
    
    create trigger check_user_regular
    after insert on utilizador_regular
    for each row execute procedure check_user_regular_proc();
    \end{verbatim}
    \normalsize
    
    \Large
    \textbf{Índices}\\
    \normalsize
    \par TODO\\
    
    \Large
    \textbf{Modelo Multidimensional}\\
    \normalsize
    \par Para criar um modelo Multidimensional, usou-se um esquema em estrela, tendo as seguintes  dimensões:
    
    \begin{enumerate}[leftmargin=3\parindent]
        \item Utilizador (Quem criou a Anomalia)
        \item Tempo (Quando a Anomalia foi criada)
        \item Local (Onde a Anomalia foi criada)
        \item Língua (Em que linguagem está a Anomalia)
    \end{enumerate}

    \par De seguida apresenta-se o código usado para criar o Modelo Multidimensional
    \small \begin{verbatim}
    TODO: APENAS ESCREVER CODIGO QUANDO ESTIVER CHECKED
    \end{verbatim}\normalsize
    
    
    \Large
    \textbf{Data Analytics}\\
    \normalsize
    \par Para efetuar a análise pedida decidiu-se, por simplicidade, usar a instrução CUBE:
    \small \begin{verbatim}
    select U.tipo, L.lingua, T.dia_da_semana, count(tipo_anomalia) 
    from f_anomalia         natural join
         d_utilizador as U  natural join
         d_tempo as T       natural join 
         d_lingua as L 
    group by cube(U.tipo, L.lingua, T.dia_da_semana);
    \end{verbatim}\normalsize
\end{document}
