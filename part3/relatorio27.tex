\documentclass[12pt]{report}
\usepackage[textwidth=17cm, margin=2cm]{geometry}
\usepackage[utf8]{inputenc}
\usepackage[T1]{fontenc}
\usepackage{graphicx}
\usepackage{enumitem}
\usepackage{multicol}
\usepackage{amsmath}
\usepackage{listings}

\begin{document}

    \begin{titlepage}
        \begin{center}

            \vspace*{\fill}
            \Huge
            \textbf{Projeto de Bases de Dados - Parte 3}
            
            \vspace*{\fill}

            \Large
            \textbf{Grupo 27} \\
            89427 - Daniel Seara - 33,3\% - 6 horas \\
            89399 - Afonso Gonçalves - 33,3\% - 6 horas \\
            89496 - Marcelo Santos - 33,3\% - 6 horas \\

            \bigskip
            \textbf{Turno:} 4ª Feira 9h30 - Lab 8\\ \textbf{Professor:} Duarte Galvão
        
        \end{center}
    \end{titlepage}

    \Large
    \textbf{Criação da base de Dados:}\\
    
    \normalsize
    \vspace{2mm}

    Assumindo que se tem um utilizador configurado num servidor PSQL, para criar a base de dados deverá correr os seguintes comandos:
    \begin{verbatim}
        $ createdb translateRight
        $ psql -d translateRight
        translateRight=# \i schema.sql
    \end{verbatim}
    
    \vspace*{15mm}


    \Large 
    \textbf{Consultas SQL:}

    \normalsize
    \vspace{2mm}

    \begin{enumerate}
        \item  WITH local\_anomalia\_count(latitude, longitude, nome, anomalia\_count)\\
               AS (SELECT L.latitude, L.longitude, L.nome, COUNT(*) \\
        \hspace*{1em} FROM (local\_publico NATURAL JOIN item) AS L, incidencia as I \\
        \hspace*{1em} WHERE L.id = I.item\_id \\
        \hspace*{1em} GROUP BY (latitude, longitude)) \\
                SELECT latitude, longitude, nome \\
                FROM local\_anomalia\_count \\
                WHERE anomalia\_count = (SELECT MAX(anomalia\_count) FROM local\_anomalia\_count)
        
        \item WITH utilizador\_regular\_tmp(email, anomalia\_count) \\
              AS (SELECT I.email, COUNT(A.id) \\
              \hspace*{1em} FROM (utilizador\_regular NATURAL JOIN incidencia) AS I, anomalia A \\
              \hspace*{1em} WHERE I.anomalia\_id = A.id \\
              \hspace*{1em} AND A.tem\_anomalia\_redacao = false \\
              \hspace*{1em} AND TIMESTAMP '2019-01-01' <= A.ts \\
              \hspace*{1em} AND A.ts < TIMESTAMP '2019-07-01' \\
              \hspace*{1em} GROUP BY I.email) \\
              SELECT email\\
              FROM utilizador\_regular\_tmp \\
              WHERE anomalia\_count = (SELECT MAX(anomalia\_count) FROM utilizador\_regular\_tmp);

        \item WITH norte\_rio\_maior \\
              AS (SELECT * FROM local\_publico \\
              \hspace*{1em} WHERE latitude > 39.336775), \\
              \hspace*{1em} utilizador\_local(email, latitude, longitude) \\
              \hspace*{1em} AS (SELECT I.email, LP.latitude, LP.longitude \\
              \hspace*{1em} FROM incidencia AS I, anomalia A, (item NATURAL JOIN local\_publico) AS LP \\
              \hspace*{1em} WHERE I.item\_id = LP.id \\
              \hspace*{1em} AND I.anomalia\_id = A.id \\
              \hspace*{1em} AND extract(year FROM A.ts) = 2019) \\
              SELECT distinct email \\
              FROM utilizador\_local U1 \\
              WHERE NOT EXISTS (SELECT latitude, longitude FROM norte\_rio\_maior\\
              EXCEPT SELECT latitude, longitude FROM utilizador\_local U2 \\
              WHERE U1.email = U2.email);

        \item WITH incidencia\_qual\_sul(email, anomalia\_id)\\
              AS (SELECT I.email, I.anomalia\_id \\
              FROM (utilizador\_qualificado NATURAL JOIN incidencia) AS I, (item NATURAL JOIN local\_publico) AS LP, anomalia A \\
              \hspace*{1em} WHERE I.item\_id = LP.id \\
              \hspace*{1em} AND I.anomalia\_id = A.id \\
              \hspace*{1em} AND LP.latitude < 39.336775\\
              \hspace*{1em} AND extract(year FROM A.ts) = extract(year FROM current\_date))\\
              SELECT distinct U1.email FROM incidencia\_qual\_sul U1 \\
              WHERE EXISTS (SELECT U2.anomalia\_id \\
              FROM incidencia\_qual\_sul U2 \\
              WHERE U2.email = U1.email \\
              EXCEPT SELECT C.anomalia\_id \\
              FROM (proposta\_de\_correcao NATURAL JOIN correcao) AS C \\
              WHERE C.email = U1.email);
        
    \end{enumerate}

    \vspace*{15mm}


    \Large 
    \textbf{Arquitetura da aplicação PHP:}

    \normalsize
    \vspace{2mm}

    \hspace*{1em} A aplicação inicia no ficheiro \textit{index.html}, onde se encontram todas as opções possíveis a realizar com a base de dados. Ao clicar em qualquer um dos links,
    é redirecionado para uma página php que permite executar a ação pretendida. Por exemplo, se o objetivo for eliminar um item, do ficheiro \textit{index.hmtl} é redirecionado para 
    o ficheiro \textit{itemrem.php} onde estão listados todos os items da base de dados e pode escolher o item a eliminar ao carregar no botão correspondente. Todas as funcionalidades 
    estão implementadas em ficheiros diferentes, não havendo um "ficheiro único" que realize todas as ações possíveis.\\

    \hspace*{1em} A única funcionalidade que não redireciona para um ficheiro php é a funcionalidade de listar anomalias a (dX, dY) graus de (latitude, longitude), onde se é redirecionado para um ficheiro html (\textit{anomalialatform.html})
    para se escrever os parâmetros desejados. Depois de preencher o formulário, é redirecionado para o ficheiro \textit{anomaliathreemonth.php}. \\

    \hspace*{1em} ATENCÃO: para se ligar à base de dados, os ficheiros php usam as variáveis de ambiente POSTGRES\_USER e POSTGRES\_PASS, que devem ser definidas pelo host da base de dados. \\

    \hspace*{1em} Qualquer página da aplicação, à exceção do \textit{index.html}, possui um link para se retroceder para a página anterior.\\

    \hspace*{1em} Os ficheiros php desta aplicação seguem a seguinte estrutura:
    \begin{enumerate}
        \item ligação à base de dados (usando a função definida no ficheiro \textit{db.php})
        \item apresentação de tabelas relevantes a executar a ação pretendida (por exemplo apresentar a lista de itens quando se quer registar itens duplicados)
        \item leitura do input do utilizador, seja por meio do preenchimento de um form (ao inserir anomalias, por exemplo) ou de selecionar uma linha da tabela relevante (para eliminar locais, por exemplo)
        \item criação de uma query SQL a executar, seguido da sua "preparação" (\textit{\$db->prepare(\$sql))})
        \item execução do query e apresentação da tabela atualizada, ou de uma mensagem de confirmação
        
    \end{enumerate}

    \hspace*{1em} Todas as ações que implicam uma mudança no estado da base de dados são realizadas a partir de transações, para garantir a atomicidade das mesmas. \\

\end{document}
