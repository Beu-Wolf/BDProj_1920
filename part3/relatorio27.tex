\documentclass[12pt]{report}
\usepackage[textwidth=17cm, margin=2cm]{geometry}
\usepackage[utf8]{inputenc}
\usepackage[T1]{fontenc}
\usepackage{graphicx}
\usepackage{enumitem}
\usepackage{multicol}
\usepackage{amsmath}

\begin{document}

    \begin{titlepage}
        \begin{center}

            \vspace*{\fill}
            \Huge
            \textbf{Projeto de Bases de Dados - Parte 3}
            
            \vspace*{\fill}

            \Large
            \textbf{Grupo 27} \\
            89427 - Daniel Seara - 33,3\% - 6 horas \\
            89399 - Afonso Gonçalves - 33,3\% - 6 horas \\
            89496 - Marcelo Santos - 33,3\% - 6 horas \\

            \bigskip
            \textbf{Turno:} 4ª Feira 9h30 - Lab 8\\ \textbf{Professor:} Duarte Galvão
        
        \end{center}
    \end{titlepage}

    \Large
    \textbf{Criação da base de Dados:}\\
    
    \normalsize
    \vspace{2mm}

    Para criar a base de dados, entre num servidor PSQL e execute o comando \textbf{\textbackslash{i}} seguido do ficheiro \textit{schema.sql}.
    Este comando irá eliminar todas as possíveis referências a tabelas com o mesmo nome no servidor e criará um conjunto de novas tabelas sem dados.

    \vspace*{15mm}


    \Large 
    \textbf{Consultas SQL:}

    \normalsize
    \vspace{2mm}

    \begin{enumerate}
        \item  WITH local\_anomalia\_count(latitude, longitude, nome, anomalia\_count)\\
               AS (SELECT L.latitude, L.longitude, L.nome, COUNT(*) \\
        \hspace*{1em} FROM (local\_publico NATURAL JOIN item) AS L, incidencia as I \\
        \hspace*{1em} WHERE L.id = I.item\_id \\
        \hspace*{1em} GROUP BY (latitude, longitude)) \\
                SELECT latitude, longitude, nome \\
                FROM local\_anomalia\_count \\
                WHERE anomalia\_count = (SELECT MAX(anomalia\_count) FROM local\_anomalia\_count)
        
        \item WITH utilizador\_regular\_tmp(email, anomalia\_count) \\
              AS (SELECT I.email, COUNT(A.id) \\
              \hspace*{1em} FROM (utilizador\_regular NATURAL JOIN incidencia) AS I, anomalia A \\
              \hspace*{1em} WHERE I.anomalia\_id = A.id \\
              \hspace*{1em} AND A.tem\_anomalia\_redacao = false \\
              \hspace*{1em} AND TIMESTAMP '2019-01-01' <= A.ts \\
              \hspace*{1em} AND A.ts < TIMESTAMP '2019-07-01' \\
              \hspace*{1em} GROUP BY I.email) \\
              SELECT email\\
              FROM utilizador\_regular\_tmp \\
              WHERE anomalia\_count = (SELECT MAX(anomalia\_count) FROM utilizador\_regular\_tmp);

        \item WITH norte\_rio\_maior \\
              AS (SELECT * FROM local\_publico \\
              \hspace*{1em} WHERE latitude > 39.336775), \\
              \hspace*{1em} utilizador\_local(email, latitude, longitude) \\
              \hspace*{1em} AS (SELECT I.email, LP.latitude, LP.longitude \\
              \hspace*{1em} FROM incidencia AS I, anomalia A, (item NATURAL JOIN local\_publico) AS LP \\
              \hspace*{1em} WHERE I.item\_id = LP.id \\
              \hspace*{1em} AND I.anomalia\_id = A.id \\
              \hspace*{1em} AND extract(year FROM A.ts) = 2019) \\
              SELECT distinct email \\
              FROM utilizador\_local U1 \\
              WHERE NOT EXISTS (SELECT latitude, longitude FROM norte\_rio\_maior\\
              EXCEPT SELECT latitude, longitude FROM utilizador\_local U2 \\
              WHERE U1.email = U2.email);

        \item WITH incidencia\_qual\_sul(email, anomalia\_id)\\
              AS (SELECT I.email, I.anomalia\_id \\
              FROM (utilizador\_qualificado NATURAL JOIN incidencia) AS I, (item NATURAL JOIN local\_publico) AS LP, anomalia A \\
              \hspace*{1em} WHERE I.item\_id = LP.id \\
              \hspace*{1em} AND I.anomalia\_id = A.id \\
              \hspace*{1em} AND LP.latitude < 39.336775\\
              \hspace*{1em} AND extract(year FROM A.ts) = extract(year FROM current\_date))\\
              SELECT distinct U1.email FROM incidencia\_qual\_sul U1 \\
              WHERE EXISTS (SELECT U2.anomalia\_id \\
              FROM incidencia\_qual\_sul U2 \\
              WHERE U2.email = U1.email \\
              EXCEPT SELECT C.anomalia\_id \\
              FROM (proposta\_de\_correcao NATURAL JOIN correcao) AS C \\
              WHERE C.email = U1.email);
        
    \end{enumerate}


\end{document}
